% Options for packages loaded elsewhere
\PassOptionsToPackage{unicode}{hyperref}
\PassOptionsToPackage{hyphens}{url}
%
\documentclass[
]{article}
\usepackage{amsmath,amssymb}
\usepackage{lmodern}
\usepackage{ifxetex,ifluatex}
\ifnum 0\ifxetex 1\fi\ifluatex 1\fi=0 % if pdftex
  \usepackage[T1]{fontenc}
  \usepackage[utf8]{inputenc}
  \usepackage{textcomp} % provide euro and other symbols
\else % if luatex or xetex
  \usepackage{unicode-math}
  \defaultfontfeatures{Scale=MatchLowercase}
  \defaultfontfeatures[\rmfamily]{Ligatures=TeX,Scale=1}
\fi
% Use upquote if available, for straight quotes in verbatim environments
\IfFileExists{upquote.sty}{\usepackage{upquote}}{}
\IfFileExists{microtype.sty}{% use microtype if available
  \usepackage[]{microtype}
  \UseMicrotypeSet[protrusion]{basicmath} % disable protrusion for tt fonts
}{}
\makeatletter
\@ifundefined{KOMAClassName}{% if non-KOMA class
  \IfFileExists{parskip.sty}{%
    \usepackage{parskip}
  }{% else
    \setlength{\parindent}{0pt}
    \setlength{\parskip}{6pt plus 2pt minus 1pt}}
}{% if KOMA class
  \KOMAoptions{parskip=half}}
\makeatother
\usepackage{xcolor}
\IfFileExists{xurl.sty}{\usepackage{xurl}}{} % add URL line breaks if available
\IfFileExists{bookmark.sty}{\usepackage{bookmark}}{\usepackage{hyperref}}
\hypersetup{
  pdftitle={Práctica 0. FMAD 2021-2022},
  pdfauthor={Gutiérrez García, Laura},
  hidelinks,
  pdfcreator={LaTeX via pandoc}}
\urlstyle{same} % disable monospaced font for URLs
\usepackage[margin=1in]{geometry}
\usepackage{color}
\usepackage{fancyvrb}
\newcommand{\VerbBar}{|}
\newcommand{\VERB}{\Verb[commandchars=\\\{\}]}
\DefineVerbatimEnvironment{Highlighting}{Verbatim}{commandchars=\\\{\}}
% Add ',fontsize=\small' for more characters per line
\usepackage{framed}
\definecolor{shadecolor}{RGB}{248,248,248}
\newenvironment{Shaded}{\begin{snugshade}}{\end{snugshade}}
\newcommand{\AlertTok}[1]{\textcolor[rgb]{0.94,0.16,0.16}{#1}}
\newcommand{\AnnotationTok}[1]{\textcolor[rgb]{0.56,0.35,0.01}{\textbf{\textit{#1}}}}
\newcommand{\AttributeTok}[1]{\textcolor[rgb]{0.77,0.63,0.00}{#1}}
\newcommand{\BaseNTok}[1]{\textcolor[rgb]{0.00,0.00,0.81}{#1}}
\newcommand{\BuiltInTok}[1]{#1}
\newcommand{\CharTok}[1]{\textcolor[rgb]{0.31,0.60,0.02}{#1}}
\newcommand{\CommentTok}[1]{\textcolor[rgb]{0.56,0.35,0.01}{\textit{#1}}}
\newcommand{\CommentVarTok}[1]{\textcolor[rgb]{0.56,0.35,0.01}{\textbf{\textit{#1}}}}
\newcommand{\ConstantTok}[1]{\textcolor[rgb]{0.00,0.00,0.00}{#1}}
\newcommand{\ControlFlowTok}[1]{\textcolor[rgb]{0.13,0.29,0.53}{\textbf{#1}}}
\newcommand{\DataTypeTok}[1]{\textcolor[rgb]{0.13,0.29,0.53}{#1}}
\newcommand{\DecValTok}[1]{\textcolor[rgb]{0.00,0.00,0.81}{#1}}
\newcommand{\DocumentationTok}[1]{\textcolor[rgb]{0.56,0.35,0.01}{\textbf{\textit{#1}}}}
\newcommand{\ErrorTok}[1]{\textcolor[rgb]{0.64,0.00,0.00}{\textbf{#1}}}
\newcommand{\ExtensionTok}[1]{#1}
\newcommand{\FloatTok}[1]{\textcolor[rgb]{0.00,0.00,0.81}{#1}}
\newcommand{\FunctionTok}[1]{\textcolor[rgb]{0.00,0.00,0.00}{#1}}
\newcommand{\ImportTok}[1]{#1}
\newcommand{\InformationTok}[1]{\textcolor[rgb]{0.56,0.35,0.01}{\textbf{\textit{#1}}}}
\newcommand{\KeywordTok}[1]{\textcolor[rgb]{0.13,0.29,0.53}{\textbf{#1}}}
\newcommand{\NormalTok}[1]{#1}
\newcommand{\OperatorTok}[1]{\textcolor[rgb]{0.81,0.36,0.00}{\textbf{#1}}}
\newcommand{\OtherTok}[1]{\textcolor[rgb]{0.56,0.35,0.01}{#1}}
\newcommand{\PreprocessorTok}[1]{\textcolor[rgb]{0.56,0.35,0.01}{\textit{#1}}}
\newcommand{\RegionMarkerTok}[1]{#1}
\newcommand{\SpecialCharTok}[1]{\textcolor[rgb]{0.00,0.00,0.00}{#1}}
\newcommand{\SpecialStringTok}[1]{\textcolor[rgb]{0.31,0.60,0.02}{#1}}
\newcommand{\StringTok}[1]{\textcolor[rgb]{0.31,0.60,0.02}{#1}}
\newcommand{\VariableTok}[1]{\textcolor[rgb]{0.00,0.00,0.00}{#1}}
\newcommand{\VerbatimStringTok}[1]{\textcolor[rgb]{0.31,0.60,0.02}{#1}}
\newcommand{\WarningTok}[1]{\textcolor[rgb]{0.56,0.35,0.01}{\textbf{\textit{#1}}}}
\usepackage{graphicx}
\makeatletter
\def\maxwidth{\ifdim\Gin@nat@width>\linewidth\linewidth\else\Gin@nat@width\fi}
\def\maxheight{\ifdim\Gin@nat@height>\textheight\textheight\else\Gin@nat@height\fi}
\makeatother
% Scale images if necessary, so that they will not overflow the page
% margins by default, and it is still possible to overwrite the defaults
% using explicit options in \includegraphics[width, height, ...]{}
\setkeys{Gin}{width=\maxwidth,height=\maxheight,keepaspectratio}
% Set default figure placement to htbp
\makeatletter
\def\fps@figure{htbp}
\makeatother
\setlength{\emergencystretch}{3em} % prevent overfull lines
\providecommand{\tightlist}{%
  \setlength{\itemsep}{0pt}\setlength{\parskip}{0pt}}
\setcounter{secnumdepth}{-\maxdimen} % remove section numbering
\ifluatex
  \usepackage{selnolig}  % disable illegal ligatures
\fi

\title{Práctica 0. FMAD 2021-2022}
\usepackage{etoolbox}
\makeatletter
\providecommand{\subtitle}[1]{% add subtitle to \maketitle
  \apptocmd{\@title}{\par {\large #1 \par}}{}{}
}
\makeatother
\subtitle{ICAI. Master en Big Data. Fundamentos Matemáticos del Análisis
de Datos (FMAD).}
\author{Gutiérrez García, Laura}
\date{Curso 2021-22. Última actualización: 2021-09-14}

\begin{document}
\maketitle

\hypertarget{ejercicio-0-ejemplo.}{%
\section{Ejercicio 0 (ejemplo).}\label{ejercicio-0-ejemplo.}}

\textbf{Enunciado:} Usa la función \texttt{seq} de R para fabricar un
vector \texttt{v} con los múltiplos de 3 del 0 al 300. Muestra los
primeros 20 elementos de \texttt{v} usando \texttt{head} y calcula:

\begin{itemize}
\tightlist
\item
  la suma del vector \texttt{v},
\item
  su media,
\item
  y su longitud.
\end{itemize}

\textbf{Respuesta:}

\begin{Shaded}
\begin{Highlighting}[]
\NormalTok{v }\OtherTok{=} \FunctionTok{seq}\NormalTok{(}\AttributeTok{from =} \DecValTok{0}\NormalTok{, }\AttributeTok{to =} \DecValTok{300}\NormalTok{, }\AttributeTok{by =} \DecValTok{3}\NormalTok{)}
\FunctionTok{head}\NormalTok{(v, }\DecValTok{20}\NormalTok{)}
\end{Highlighting}
\end{Shaded}

\begin{verbatim}
##  [1]  0  3  6  9 12 15 18 21 24 27 30 33 36 39 42 45 48 51 54 57
\end{verbatim}

Suma de \texttt{v}

\begin{Shaded}
\begin{Highlighting}[]
\FunctionTok{sum}\NormalTok{(v)}
\end{Highlighting}
\end{Shaded}

\begin{verbatim}
## [1] 15150
\end{verbatim}

Media:

\begin{Shaded}
\begin{Highlighting}[]
\FunctionTok{mean}\NormalTok{(v)}
\end{Highlighting}
\end{Shaded}

\begin{verbatim}
## [1] 150
\end{verbatim}

Longitud:

\begin{Shaded}
\begin{Highlighting}[]
\FunctionTok{length}\NormalTok{(v)}
\end{Highlighting}
\end{Shaded}

\begin{verbatim}
## [1] 101
\end{verbatim}

\hypertarget{libreruxedas}{%
\section{Librerías}\label{libreruxedas}}

Antes de comenzar con la práctica, cargamos todas las librerías
necesarias:

\begin{Shaded}
\begin{Highlighting}[]
\FunctionTok{library}\NormalTok{(tidyverse) }\CommentTok{\# Uso de dplyr y ggplot}
\FunctionTok{library}\NormalTok{(haven) }\CommentTok{\# Lectura de datos desde stata}
\end{Highlighting}
\end{Shaded}

\hypertarget{ejercicio-1.}{%
\section{Ejercicio 1.}\label{ejercicio-1.}}

\textbf{Enunciado:} Usando la función sample crea un vector
dado\_honesto con 100 números del 1 al 6. Haz una tabla de frecuencias
absolutas (de dos maneras, con table y dplyr) y una tabla de frecuencias
relativas.

\textbf{Respuesta:} Creo el vector:

\begin{Shaded}
\begin{Highlighting}[]
\FunctionTok{set.seed}\NormalTok{(}\DecValTok{2021}\NormalTok{)}
\NormalTok{(dado\_honesto }\OtherTok{\textless{}{-}} \FunctionTok{sample}\NormalTok{(}\DecValTok{1}\SpecialCharTok{:}\DecValTok{6}\NormalTok{,}\DecValTok{100}\NormalTok{, }\AttributeTok{replace =}\NormalTok{ T))}
\end{Highlighting}
\end{Shaded}

\begin{verbatim}
##   [1] 6 6 2 4 4 6 6 3 6 6 5 1 4 3 4 2 3 4 5 3 6 2 4 5 6 2 3 4 5 6 5 1 6 2 3 3 2
##  [38] 6 6 6 2 5 6 3 2 1 1 6 5 4 4 6 3 3 2 1 2 1 1 1 1 5 3 3 1 4 6 6 6 2 1 3 4 2
##  [75] 5 2 6 6 4 6 6 3 4 5 1 6 5 3 1 5 3 1 3 6 4 6 6 5 1 3
\end{verbatim}

Tabla de frecuencias absolutas

\begin{Shaded}
\begin{Highlighting}[]
\FunctionTok{table}\NormalTok{(dado\_honesto)}
\end{Highlighting}
\end{Shaded}

\begin{verbatim}
## dado_honesto
##  1  2  3  4  5  6 
## 15 13 18 14 13 27
\end{verbatim}

\begin{Shaded}
\begin{Highlighting}[]
\CommentTok{\# Creo el dataframe para poder trabajar con dplyr}
\NormalTok{df }\OtherTok{\textless{}{-}} \FunctionTok{data.frame}\NormalTok{(}\FunctionTok{c}\NormalTok{(}\DecValTok{1}\SpecialCharTok{:}\FunctionTok{length}\NormalTok{(dado\_honesto)),dado\_honesto)}
\FunctionTok{names}\NormalTok{(df) }\OtherTok{\textless{}{-}} \FunctionTok{c}\NormalTok{(}\StringTok{"Tirada"}\NormalTok{,}\StringTok{"dado\_honesto"}\NormalTok{)}

\NormalTok{df }\SpecialCharTok{\%\textgreater{}\%} 
  \FunctionTok{count}\NormalTok{(dado\_honesto)}
\end{Highlighting}
\end{Shaded}

\begin{verbatim}
##   dado_honesto  n
## 1            1 15
## 2            2 13
## 3            3 18
## 4            4 14
## 5            5 13
## 6            6 27
\end{verbatim}

Tabla de frecuencias relativas

\begin{Shaded}
\begin{Highlighting}[]
\FunctionTok{prop.table}\NormalTok{(}\FunctionTok{table}\NormalTok{(dado\_honesto))}
\end{Highlighting}
\end{Shaded}

\begin{verbatim}
## dado_honesto
##    1    2    3    4    5    6 
## 0.15 0.13 0.18 0.14 0.13 0.27
\end{verbatim}

\begin{Shaded}
\begin{Highlighting}[]
\NormalTok{      df }\SpecialCharTok{\%\textgreater{}\%} 
        \FunctionTok{count}\NormalTok{(dado\_honesto) }\SpecialCharTok{\%\textgreater{}\%}
          \FunctionTok{mutate}\NormalTok{(dado\_honesto, }\AttributeTok{relFreq =} \FunctionTok{prop.table}\NormalTok{(n), }\AttributeTok{n=}\ConstantTok{NULL}\NormalTok{)}
\end{Highlighting}
\end{Shaded}

\begin{verbatim}
##   dado_honesto relFreq
## 1            1    0.15
## 2            2    0.13
## 3            3    0.18
## 4            4    0.14
## 5            5    0.13
## 6            6    0.27
\end{verbatim}

\hypertarget{ejercicio-2.}{%
\section{Ejercicio 2.}\label{ejercicio-2.}}

\textbf{Enunciado:} A continuación crea un nuevo vector dado\_cargado de
manera que la probabilidad de que el número elegido valga 6 sea el doble
que la probabilidad de elegir cualquiera de los cinco números restantes.
Lee la ayuda de sample si lo necesitas. De nuevo, haz tablas de
frecuencias absolutas y relativas de este segundo vector.

\textbf{Respuesta:} Creo el vector:

\begin{Shaded}
\begin{Highlighting}[]
\FunctionTok{set.seed}\NormalTok{(}\DecValTok{2021}\NormalTok{)}
\NormalTok{(dado\_cargado }\OtherTok{\textless{}{-}} \FunctionTok{sample}\NormalTok{(}\DecValTok{1}\SpecialCharTok{:}\DecValTok{6}\NormalTok{,}\DecValTok{100}\NormalTok{, }\AttributeTok{replace =}\NormalTok{ T, }\AttributeTok{prob =} \FunctionTok{rep}\NormalTok{( }\FunctionTok{c}\NormalTok{(}\DecValTok{1}\SpecialCharTok{/}\DecValTok{7}\NormalTok{,}\DecValTok{2}\SpecialCharTok{/}\DecValTok{7}\NormalTok{),}\FunctionTok{c}\NormalTok{(}\DecValTok{5}\NormalTok{,}\DecValTok{1}\NormalTok{))))}
\end{Highlighting}
\end{Shaded}

\begin{verbatim}
##   [1] 4 2 5 3 5 5 5 6 2 1 6 2 5 4 2 6 4 1 1 4 6 1 3 6 5 1 1 1 1 2 6 1 1 1 4 6 1
##  [38] 6 1 4 6 2 3 1 6 2 1 6 6 2 4 5 6 4 3 6 4 3 1 2 6 4 6 3 2 6 6 5 6 6 6 5 6 1
##  [75] 6 4 4 3 5 1 4 3 1 5 3 6 1 2 4 3 6 5 1 4 6 1 5 2 2 5
\end{verbatim}

Tabla de frecuencias absolutas

\begin{Shaded}
\begin{Highlighting}[]
\FunctionTok{table}\NormalTok{(dado\_cargado)}
\end{Highlighting}
\end{Shaded}

\begin{verbatim}
## dado_cargado
##  1  2  3  4  5  6 
## 22 13 10 15 14 26
\end{verbatim}

\begin{Shaded}
\begin{Highlighting}[]
\CommentTok{\# Creo el dataframe para poder trabajar con dplyr}
\NormalTok{df2 }\OtherTok{\textless{}{-}} \FunctionTok{data.frame}\NormalTok{(}\FunctionTok{c}\NormalTok{(}\DecValTok{1}\SpecialCharTok{:}\FunctionTok{length}\NormalTok{(dado\_cargado)),dado\_cargado)}
\FunctionTok{names}\NormalTok{(df) }\OtherTok{\textless{}{-}} \FunctionTok{c}\NormalTok{(}\StringTok{"Tirada"}\NormalTok{,}\StringTok{"dado\_cargado"}\NormalTok{)}

\NormalTok{df2 }\SpecialCharTok{\%\textgreater{}\%} 
  \FunctionTok{count}\NormalTok{(dado\_cargado)}
\end{Highlighting}
\end{Shaded}

\begin{verbatim}
##   dado_cargado  n
## 1            1 22
## 2            2 13
## 3            3 10
## 4            4 15
## 5            5 14
## 6            6 26
\end{verbatim}

Tabla de frecuencias relativas

\begin{Shaded}
\begin{Highlighting}[]
\FunctionTok{prop.table}\NormalTok{(}\FunctionTok{table}\NormalTok{(dado\_cargado))}
\end{Highlighting}
\end{Shaded}

\begin{verbatim}
## dado_cargado
##    1    2    3    4    5    6 
## 0.22 0.13 0.10 0.15 0.14 0.26
\end{verbatim}

\begin{Shaded}
\begin{Highlighting}[]
\NormalTok{      df2 }\SpecialCharTok{\%\textgreater{}\%} 
        \FunctionTok{count}\NormalTok{(dado\_cargado) }\SpecialCharTok{\%\textgreater{}\%}
          \FunctionTok{mutate}\NormalTok{(dado\_cargado, }\AttributeTok{relFreq =} \FunctionTok{prop.table}\NormalTok{(n), }\AttributeTok{n=}\ConstantTok{NULL}\NormalTok{)}
\end{Highlighting}
\end{Shaded}

\begin{verbatim}
##   dado_cargado relFreq
## 1            1    0.22
## 2            2    0.13
## 3            3    0.10
## 4            4    0.15
## 5            5    0.14
## 6            6    0.26
\end{verbatim}

\hypertarget{ejercicio-3.}{%
\section{Ejercicio 3.}\label{ejercicio-3.}}

\textbf{Enunciado:} Utiliza las funciones rep y seq para crear tres
vectores v1, v2 y v3 con estos elementos respectivamente:

4, 4, 4, 4, 3, 3, 3, 3, 2, 2, 2, 2, 1, 1, 1, 1

1, 2, 2, 3, 3, 3, 4, 4, 4, 4, 5, 5, 5, 5, 5

1, 2, 3, 4, 1, 2, 3, 4, 1, 2, 3, 4, 1, 2, 3, 4

\textbf{Respuesta:} Creo los vectores:

\begin{Shaded}
\begin{Highlighting}[]
\NormalTok{(v1 }\OtherTok{\textless{}{-}} \FunctionTok{rep}\NormalTok{(}\FunctionTok{seq}\NormalTok{(}\DecValTok{4}\NormalTok{,}\DecValTok{1}\NormalTok{), }\AttributeTok{each =} \DecValTok{4}\NormalTok{))}
\end{Highlighting}
\end{Shaded}

\begin{verbatim}
##  [1] 4 4 4 4 3 3 3 3 2 2 2 2 1 1 1 1
\end{verbatim}

\begin{Shaded}
\begin{Highlighting}[]
\NormalTok{(v2 }\OtherTok{\textless{}{-}} \FunctionTok{rep}\NormalTok{(}\FunctionTok{seq}\NormalTok{(}\DecValTok{1}\NormalTok{,}\DecValTok{5}\NormalTok{), }\AttributeTok{times =} \DecValTok{1}\SpecialCharTok{:}\DecValTok{5}\NormalTok{))}
\end{Highlighting}
\end{Shaded}

\begin{verbatim}
##  [1] 1 2 2 3 3 3 4 4 4 4 5 5 5 5 5
\end{verbatim}

\begin{Shaded}
\begin{Highlighting}[]
\NormalTok{(v3 }\OtherTok{\textless{}{-}} \FunctionTok{rep}\NormalTok{(}\FunctionTok{seq}\NormalTok{(}\DecValTok{4}\NormalTok{,}\DecValTok{1}\NormalTok{), }\AttributeTok{times =} \DecValTok{4}\NormalTok{))}
\end{Highlighting}
\end{Shaded}

\begin{verbatim}
##  [1] 4 3 2 1 4 3 2 1 4 3 2 1 4 3 2 1
\end{verbatim}

\hypertarget{ejercicio-4.}{%
\section{Ejercicio 4.}\label{ejercicio-4.}}

\textbf{Enunciado:} Utilizando la tabla mpg de la librería tidyverse
crea una tabla mpg2 que:

\begin{enumerate}
\def\labelenumi{\Roman{enumi})}
\item
  contenga las filas en las que la variable class toma el valor pickup.
\item
  y las columnas de la tabla original cuyos nombres empiezan por c.~No
  se trata de que las selecciones a mano, por sus nombres
\end{enumerate}

\textbf{Respuesta:} Hago la selección:

\begin{Shaded}
\begin{Highlighting}[]
\NormalTok{mpg }\SpecialCharTok{\%\textgreater{}\%} 
  \FunctionTok{filter}\NormalTok{(class }\SpecialCharTok{==} \StringTok{"pickup"}\NormalTok{) }\SpecialCharTok{\%\textgreater{}\%} 
  \FunctionTok{select}\NormalTok{(}\FunctionTok{starts\_with}\NormalTok{(}\StringTok{"c"}\NormalTok{))  }
\end{Highlighting}
\end{Shaded}

\begin{verbatim}
## # A tibble: 33 x 3
##      cyl   cty class 
##    <int> <int> <chr> 
##  1     6    15 pickup
##  2     6    14 pickup
##  3     6    13 pickup
##  4     6    14 pickup
##  5     8    14 pickup
##  6     8    14 pickup
##  7     8     9 pickup
##  8     8    11 pickup
##  9     8    11 pickup
## 10     8    12 pickup
## # ... with 23 more rows
\end{verbatim}

\hypertarget{ejercicio-5.}{%
\section{Ejercicio 5.}\label{ejercicio-5.}}

\textbf{Enunciado:} Descarga el fichero census.dta. Averigua de qué tipo
de fichero se trata y usa la herramienta Import DataSet del panel
Environment de RStudio para leer con R los datos de ese fichero.
Asegúrate de copiar en esta práctica los dos primeros comandos que
llevan a cabo la importación (excluye el comando View) y que descubrirás
al usar esa herramienta. Después completa los siguientes apartados con
esos datos y usando dplyr y ggplot:

\textbf{Respuesta:} Importo datos stata:

\begin{Shaded}
\begin{Highlighting}[]
\NormalTok{census }\OtherTok{\textless{}{-}} \FunctionTok{read\_dta}\NormalTok{(}\StringTok{"census.dta"}\NormalTok{)}
\end{Highlighting}
\end{Shaded}

¿Cuáles son las poblaciones totales de las regiones censales?

\begin{Shaded}
\begin{Highlighting}[]
\NormalTok{census }\SpecialCharTok{\%\textgreater{}\%} 
    \FunctionTok{group\_by}\NormalTok{(region)}\SpecialCharTok{\%\textgreater{}\%} 
  \FunctionTok{summarise}\NormalTok{(}\AttributeTok{Total =} \FunctionTok{sum}\NormalTok{(pop))}
\end{Highlighting}
\end{Shaded}

\begin{verbatim}
## # A tibble: 4 x 2
##        region    Total
##     <dbl+lbl>    <dbl>
## 1 1 [NE]      49135283
## 2 2 [N Cntrl] 58865670
## 3 3 [South]   74734029
## 4 4 [West]    43172490
\end{verbatim}

Representa esas poblaciones totales en un diagrama de barras (una barra
por región censal).

\begin{Shaded}
\begin{Highlighting}[]
\NormalTok{pobl }\OtherTok{\textless{}{-}}\NormalTok{ census }\SpecialCharTok{\%\textgreater{}\%} 
         \FunctionTok{group\_by}\NormalTok{(region)}\SpecialCharTok{\%\textgreater{}\%} 
          \FunctionTok{summarise}\NormalTok{(}\AttributeTok{Total =} \FunctionTok{sum}\NormalTok{(pop))}
\NormalTok{pobl}\SpecialCharTok{$}\NormalTok{region }\OtherTok{\textless{}{-}} \FunctionTok{as.factor}\NormalTok{(pobl}\SpecialCharTok{$}\NormalTok{region)}

\CommentTok{\# Barplot}
\FunctionTok{ggplot}\NormalTok{(pobl, }\FunctionTok{aes}\NormalTok{(}\AttributeTok{x=}\NormalTok{region, }\AttributeTok{y=}\NormalTok{Total, }\AttributeTok{fill=}\NormalTok{region)) }\SpecialCharTok{+} 
  \FunctionTok{geom\_bar}\NormalTok{(}\AttributeTok{stat =} \StringTok{"identity"}\NormalTok{)}
\end{Highlighting}
\end{Shaded}

\includegraphics{P00_GutierrezGarcia_Laura_files/figure-latex/unnamed-chunk-16-1.pdf}

Ordena los estados por población, de mayor a menor.

\begin{Shaded}
\begin{Highlighting}[]
\CommentTok{\# Con R}
\NormalTok{census}\SpecialCharTok{$}\NormalTok{state[}\FunctionTok{order}\NormalTok{(census}\SpecialCharTok{$}\NormalTok{pop,}\AttributeTok{decreasing =} \ConstantTok{TRUE}\NormalTok{)]}
\end{Highlighting}
\end{Shaded}

\begin{verbatim}
##  [1] "California"    "New York"      "Texas"         "Pennsylvania" 
##  [5] "Illinois"      "Ohio"          "Florida"       "Michigan"     
##  [9] "New Jersey"    "N. Carolina"   "Massachusetts" "Indiana"      
## [13] "Georgia"       "Virginia"      "Missouri"      "Wisconsin"    
## [17] "Tennessee"     "Maryland"      "Louisiana"     "Washington"   
## [21] "Minnesota"     "Alabama"       "Kentucky"      "S. Carolina"  
## [25] "Connecticut"   "Oklahoma"      "Iowa"          "Colorado"     
## [29] "Arizona"       "Oregon"        "Mississippi"   "Kansas"       
## [33] "Arkansas"      "W. Virginia"   "Nebraska"      "Utah"         
## [37] "New Mexico"    "Maine"         "Hawaii"        "Rhode Island" 
## [41] "Idaho"         "New Hampshire" "Nevada"        "Montana"      
## [45] "S. Dakota"     "N. Dakota"     "Delaware"      "Vermont"      
## [49] "Wyoming"       "Alaska"
\end{verbatim}

\begin{Shaded}
\begin{Highlighting}[]
\CommentTok{\# Con dplyr}
\NormalTok{census }\SpecialCharTok{\%\textgreater{}\%} 
  \FunctionTok{select}\NormalTok{(state,pop) }\SpecialCharTok{\%\textgreater{}\%} 
  \FunctionTok{arrange}\NormalTok{(}\FunctionTok{across}\NormalTok{(pop,desc))}
\end{Highlighting}
\end{Shaded}

\begin{verbatim}
## # A tibble: 50 x 2
##    state             pop
##    <chr>           <dbl>
##  1 California   23667902
##  2 New York     17558072
##  3 Texas        14229191
##  4 Pennsylvania 11863895
##  5 Illinois     11426518
##  6 Ohio         10797630
##  7 Florida       9746324
##  8 Michigan      9262078
##  9 New Jersey    7364823
## 10 N. Carolina   5881766
## # ... with 40 more rows
\end{verbatim}

Crea una nueva variable que contenga la tasa de divorcios /matrimonios
para cada estado y después muestro la tasa por estado.

\begin{Shaded}
\begin{Highlighting}[]
\NormalTok{census }\OtherTok{\textless{}{-}}\NormalTok{ census }\SpecialCharTok{\%\textgreater{}\%} 
  \FunctionTok{mutate}\NormalTok{(}\AttributeTok{tasa=}\NormalTok{ divorce}\SpecialCharTok{/}\NormalTok{marriage) }

\NormalTok{census }\SpecialCharTok{\%\textgreater{}\%} 
  \FunctionTok{select}\NormalTok{(state,tasa)}
\end{Highlighting}
\end{Shaded}

\begin{verbatim}
## # A tibble: 50 x 2
##    state        tasa
##    <chr>       <dbl>
##  1 Alabama     0.546
##  2 Alaska      0.656
##  3 Arizona     0.659
##  4 Arkansas    0.599
##  5 California  0.633
##  6 Colorado    0.532
##  7 Connecticut 0.518
##  8 Delaware    0.521
##  9 Florida     0.661
## 10 Georgia     0.492
## # ... with 40 more rows
\end{verbatim}

Si nos preguntamos cuáles son los estados más envejecidos podemos
responder de dos maneras. Mirando la edad mediana o mirando en qué
estados la franja de mayor edad representa una proporción más alta de la
población total. Haz una tabla en la que aparezcan los valores de estos
dos criterios, ordenada según la edad mediana decreciente y muestra los
10 primeros estados de esa tabla.

En primer lugar, calculamos la proporción y la guardamos como variable.
Posteriormente, seleccionamos las columnas de interés (estado, edad
mediana y proporción) y las mostramos según el orden descendente para la
variable de la edad mediana:

\begin{Shaded}
\begin{Highlighting}[]
\NormalTok{census }\SpecialCharTok{\%\textgreater{}\%} 
  \FunctionTok{mutate}\NormalTok{(}\AttributeTok{prop65=}\NormalTok{ pop65p}\SpecialCharTok{/}\NormalTok{pop) }\SpecialCharTok{\%\textgreater{}\%} 
  \FunctionTok{select}\NormalTok{(state,medage,prop65) }\SpecialCharTok{\%\textgreater{}\%} 
  \FunctionTok{arrange}\NormalTok{(}\FunctionTok{desc}\NormalTok{(medage)) }\SpecialCharTok{\%\textgreater{}\%} 
  \FunctionTok{head}\NormalTok{(}\DecValTok{10}\NormalTok{)}
\end{Highlighting}
\end{Shaded}

\begin{verbatim}
## # A tibble: 10 x 3
##    state         medage prop65
##    <chr>          <dbl>  <dbl>
##  1 Florida         34.7  0.173
##  2 New Jersey      32.2  0.117
##  3 Pennsylvania    32.1  0.129
##  4 Connecticut     32    0.117
##  5 New York        31.9  0.123
##  6 Rhode Island    31.8  0.134
##  7 Massachusetts   31.2  0.127
##  8 Missouri        30.9  0.132
##  9 Arkansas        30.6  0.137
## 10 Maine           30.4  0.125
\end{verbatim}

Haz un histograma (con 10 intervalos) de los valores de la variable
medage (edad mediana) y con la curva de densidad de la variable
superpuesta.

\begin{Shaded}
\begin{Highlighting}[]
\FunctionTok{ggplot}\NormalTok{(census, }\FunctionTok{aes}\NormalTok{(}\AttributeTok{x =}\NormalTok{ medage)) }\SpecialCharTok{+} 
  \FunctionTok{geom\_histogram}\NormalTok{(}\FunctionTok{aes}\NormalTok{(}\AttributeTok{y=}\FunctionTok{stat}\NormalTok{(density)), }
                 \AttributeTok{bins=}\DecValTok{10}\NormalTok{, }\AttributeTok{fill =} \StringTok{"orange"}\NormalTok{, }\AttributeTok{color=}\StringTok{"black"}\NormalTok{)  }\SpecialCharTok{+} 
  \FunctionTok{geom\_density}\NormalTok{(}\AttributeTok{color=}\StringTok{"red"}\NormalTok{, }\AttributeTok{size=}\FloatTok{1.5}\NormalTok{)}
\end{Highlighting}
\end{Shaded}

\includegraphics{P00_GutierrezGarcia_Laura_files/figure-latex/unnamed-chunk-20-1.pdf}

\end{document}
